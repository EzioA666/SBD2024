\documentclass{article}

% packages for math
\usepackage{amsthm, amsmath, amssymb, amsfonts}

% package for including images
\usepackage{graphicx}

% environment for solutions
\theoremstyle{remark} \newtheorem*{solution}{Solution}

% capital letters for problem parts
\renewcommand{\theenumi}{\Alph{enumi}}

% no page numbers
\pagenumbering{gobble}

% codeblock
\usepackage{listings}
\usepackage{xcolor}

\definecolor{codegreen}{rgb}{0,0.6,0}
\definecolor{codegray}{rgb}{0.5,0.5,0.5}
\definecolor{codepurple}{rgb}{0.58,0,0.82}
\definecolor{backcolour}{rgb}{0.95,0.95,0.95}

\lstdefinestyle{mystyle}{
    backgroundcolor=\color{backcolour},
    commentstyle=\color{codegreen},
    keywordstyle=[1]{\color{blue}},
    keywordstyle=[2]{\color{blue}},
    numberstyle=\tiny\color{codegray},
    stringstyle=\color{codepurple},
    basicstyle=\ttfamily\footnotesize,
    breakatwhitespace=false,
    breaklines=true,
    captionpos=b,
    keepspaces=true,
    showspaces=false,
    showstringspaces=false,
    showtabs=false,
    tabsize=2
}

\lstset{style=mystyle}

\usepackage{syntax}
\setlength{\grammarindent}{6em}

\usepackage{mdframed}

\title{
Assignment 6
}
\author{CAS CS 320: Principles of Programming Languages}
\date{Due: \textbf{Friday March 22, 2024 by 11:59PM}}

\begin{document}
\maketitle
\section*{Submission Instructions}
\begin{itemize}
\item
You will submit a \texttt{.pdf} of your solutions on Gradescope.  Your
solutions must be legible. If you do not have neat handwriting, please
type your solutions.
\item
\textbf{Put a box around the final answer in your solution.}  Or
otherwise make the final answer in your solution abundantly clear.
\item
Choose the correct pages corresponding to each problem in Gradescope.
Note that Gradescope registers your submission as soon as you submit
it, so you don’t need to rush to choose corresponding pages.
\textbf{For multipart questions, please make sure each part is
  accounted for.}
\end{itemize}
We will dock points if any of these instructions are not followed.

\pagebreak
\section{A Simple Ambiguous Grammar}

\begin{mdframed}
\begin{grammar}
<start> ::= \lit*{A} <start> \lit*{B}
\alt <a>

<a> ::= \lit*{A} <a>
\alt \lit*{A} <b>

<b> ::= \lit*{B} \lit*{B} <b>
\alt \lit*{B} \lit*{B}

\end{grammar}
\end{mdframed}

\begin{enumerate}
\item
Find the \textit{shortest} sentence which has multiple
derivations. You do not need to write down the derivations.
\item
Write down a right linear regular grammar for the set of sentences of
the above grammar which have \textit{at least 2 \texttt{A} symbols and
  at least 2 \texttt{B} symbols.}
\item
Write down a regular expression for the set of sentences of the above
grammar.

\end{enumerate}


\pagebreak
\section{A More Interesting Ambiguous Grammar}

Consider the following EBNF grammar. Note that there are two uses of
parentheses, one which is part of the language and one which is part
of EBNF syntax.  Also note the use of the visible space symbol
`\textvisiblespace' which indicates that a space \textit{must} appear
in the sentence (whereas we typically use whitespace for convenience to visually separate terminal symbols).

\begin{mdframed}
\begin{grammar}
<prgm> ::= $\{$ \lit*{LET} <idnt> \lit*{=} <expr> $\}$

<expr> ::=
\lit*{FUN} <idnt> \lit*{=>} <expr>
\alt \lit*{CASE} <expr> \lit*{OF} $\{$ \lit*{/} <pat> \lit*{->} <expr> $\}$
\alt <term> $\{$ ( \lit*{+} | \lit*{-} ) <term>$\}$

<term> ::= (<idnt> | <num> | \lit*{(} <expr> \lit*{)}) [ \lit*{\textvisiblespace} <term> ]

<pat> ::= <idnt> | <num> | \lit*{*}

<num> ::= (\lit*{0} | \lit*{1} | \lit*{2} | $\dots$ | \lit*{9}) $\{$ \lit*{0} | \lit*{1} | \lit*{2} | $\dots$ | \lit*{9} $\}$

<idnt> ::= (\lit*{a} | \lit*{b} | \lit*{c} | $\dots$ | \lit*{z}) $\{$ \lit*{a} | \lit*{b} | \lit*{c} | $\dots$ | \lit*{z} $\}$

\end{grammar}
\end{mdframed}
\begin{enumerate}
\item
\textbf{TRUE} or \textbf{FALSE}, the following program has a
derivation in the above grammar:
\begin{lstlisting}[mathescape=true]
LET fib = FUN x =>
  CASE x OF
  / 0 -> 0
  / 1 -> 1
  / n -> fib$\text{\textvisiblespace}$(n - 1) + fib$\text{\textvisiblespace}$(n - 2)
\end{lstlisting}
\pagebreak
\item Give a derivation for the following program.
\begin{lstlisting}
LET z = FUN x =>
  CASE x OF
  / 0 -> 1
  / * -> 0
\end{lstlisting}
\textit{Notes.}
\begin{itemize}
\item
The newlines are for visual convenience, your derivation should end in
the single line
\begin{lstlisting}
LET z = FUN x => CASE x OF / 0 -> 1 / * -> 0
\end{lstlisting}
\item
The sentential form may become somewhat wide. You may shorten terminal
and nonterminal symbols if it is sufficiently clear.
\item
You may replace nonterminal symbols in parallel, as in the previous
assignment.
\item
Do not include EBNF syntax in the derivation, you should immediately
replace the EBNF syntax with a corresponding sentential form, e.g.,
\begin{lstlisting}
<num>  ==> 00120

<expr> ==>
CASE <expr> OF / <pat> -> <expr> / <pat> -> <expr>
\end{lstlisting}
\end{itemize}
\pagebreak
\item
Rewrite the production rule for \texttt{<term>} using standard BNF
syntax.  You may not introduce new nonterminal symbols.
\item
Demonstrate that this grammar is ambiguous by finding a sentence which
has multiple derivations.  You do not need to give the derivations.
\item
Change the above grammar by introducing a single terminal symbol so
that it is no longer ambiguous.  You don't need to reproduce the
entire grammar, just give the production rule where you put the
terminal symbol.
\end{enumerate}


\pagebreak
\section{Regular Grammars and Expressions}
\begin{enumerate}
\item Write down a regular expression which recognizes all binary strings without a contiguous subsequence of \texttt{1}s of length $3$, e.g., $\epsilon$ and \texttt{001} and \texttt{00110001001000001} but not \texttt{111} or \texttt{0011000101111001}.
\item Write down a right linear regular grammar which recognizes all binary strings with an odd number of zeros and an odd number of ones, e.g., \texttt{01} and \texttt{001011} but not \texttt{001100010} or \texttt{0011} (\textit{Note.} It would be a fair amount more difficult to give a regular expression, even though the two representations are equivalent in expressivity).
\item Write down a regular expression which recognizes all strings of characters without whitespace or parentheses which represent (possibly empty) \texttt{int list}'s of \texttt{0}'s in OCaml, e.g., \texttt{0::0::[]} and \texttt{{0::[0;0;0]}} and \texttt{[0;]} (note the optional trailing semicolon) and \texttt{[]}.
\end{enumerate}


\end{document}
